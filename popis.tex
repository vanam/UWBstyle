
\chap Vlastnosti šablony

\sec Konzervativní titulní strana

Dle požadavku katedry jsem vytvořil konzervativní variantu titulní strany, na které se nenachází ani logo ani další ozdobné prvky. Použití je jednoduché -- stačí nadefinovat makro "\def\conservative{}" a vyplnit konzervativní titulek "\conservativetitle{\the\title: \the\subtitle}".


\secc Minimální dokument s konzervativní titulní stranou

\begtt
\def\conservative{}

\input uwbstyle

\worktype [B/CZ]
\faculty {FAV}
\department {Katedra permoníků}
\title {Minimální dokument}
\conservativetitle{\the\title}
\author {Pepa z Depa}
\date {Leden 2013}
\abstractEN {This document is for testing purpose only.}
\abstractCZ {Tento dokument je pouze pro potřeby testování.}
\declaration {Prohlašuji, že jsem se neflákal.}
\makefront

\chap Úvod

Text úvodu.

\sec Myšlenka

Další text.
\bye
\endtt

\sec Drobné typografické změny

\begitems

* {\bf Loga a barvy fakult} -- Barva ozdobných prvků šablony a logo se mění dle použité fakulty a vychází z {\sl Jednotného vizuálního vzhledu} univerzity\fnote{\url{https://www.zcu.cz/about/vyznamne-dokumenty/Manual_jednotneho_vizualniho_stylu.pdf}}.

* {\bf Kapitoly a přílohy} -- Kapitoly a přílohy začínají vždy na liché stránce.

* {\bf Poznámky pod čarou} -- Poznámky pod čarou mají odlehčenější značku v textu\fnote{Ukázka poznámky pod čarou.}.

\enditems
